\documentclass[11pt,a4paper]{article}
% 使用ctexart文档类,自动配置中文字体
\usepackage[UTF8]{ctex}

% \usepackage{xeCJK}
% 设置字体
\setCJKmainfont{SimSun}[AutoFakeBold=2.5]  % 宋体
% \setCJKmainfont[AutoFakeBold=1.3, ItalicFont={KaiTi}]{FangSong}
% \setCJKmainfont[AutoFakeBold=4,AutoFakeSlant=0.2]{SimSun}
\setCJKsansfont{Microsoft YaHei}  % 微软雅黑
\setmainfont{Times New Roman}  % 英文衬线字体
\setsansfont{Arial}  % 英文无衬线字体
\newCJKfontfamily\cjkit{KaiTi}[AutoFakeBold=2,AutoFakeSlant=0.2]
% 基础包
\usepackage{hyperref}
\usepackage[margin=0.75in]{geometry}  % 调整边距
\usepackage{enumitem}
\usepackage{fontawesome5}
\usepackage{color}
\usepackage{array}
\usepackage{tabularx}
\usepackage{ragged2e}  % 用于文本对齐
\usepackage{amsmath}
\usepackage{amssymb}
\usepackage{CJKfntef}
\usepackage[chinese, provide=*]{babel}
\usepackage{hologo}
\usepackage{metalogo}
\usepackage{amsmath} % 导入amsmath包以使用\hdashline命令
% \babelfont{rm}{FandolSong}
% 自定义颜色
\definecolor{linkcolor}{RGB}{0,122,255}     % 链接蓝
\definecolor{graytext}{RGB}{100,100,100}    % 调整副文本灰色
\definecolor{sectioncolor}{RGB}{40,40,40}   % 调整标题颜色
% 自定义链接样式
\hypersetup{
    colorlinks=true,
    urlcolor=linkcolor,
    linkcolor=linkcolor
}
% 自定义列表样式
\setlist[itemize]{
    leftmargin=*,
    nosep,
    topsep=0.1em,     % 减小顶部间距
    parsep=0.2em,     % 减小段落间距
    itemsep=0.2em,    % 减小项目间距
    before={\vspace{0.2em}},  % 减小列表前间距
    after={\vspace{0.4em}}    % 减小列表后间距
}
% 调整全局行间距
\renewcommand{\baselinestretch}{0.95}  % 减小行间距
% 重定义section命令
\renewcommand{\section}[1]{%
    \vspace{0.6em}%    % 减小节前间距
    {\sffamily\Large\bfseries\color{sectioncolor}#1}%
    \vspace{0.2em}%    % 减小标题和分隔线间距
    \hrule height 0.5pt
    \vspace{0.4em}%    % 减小分隔线和内容间距
}
\newcommand{\arrowsymbol}{\ensuremath{\Rightarrow}}
% 调整标题和联系方式的间距
\newcommand{\name}[1]{{\sffamily\huge\bfseries #1}\\[- 0.2em]}
% 调整公司信息和职位的间距
\newcommand{\company}[1]{%
    {\vspace{0.2em}\sffamily\textbf{\large#1}}%
      % 增加公司信息后的垂直间距
}
\newcommand{\position}[2]{%
    \vspace{0.1em}  % 减小与公司名的间距
    \role{#1}\hfill\role{#2}%
    \vspace{0.2em}  % 减小与列表的间距
}
% 定义个人信息表格列格式
\newcolumntype{L}{>{\RaggedRight\arraybackslash}X}
% 自定义命令
\newcommand{\role}[1]{{\sffamily\color{graytext}\small#1}}
\newcommand{\daterange}[1]{\hfill{\sffamily\color{graytext}\small#1}}
\newcommand{\skill}[1]{\textbf{#1}}
\newcommand{\project}[1]{{\sffamily\textbf{\cjkit { \textit {\itshape#1}}}}}
\newcommand{\separator}{\textbullet~}  % 分隔符

\begin{document}

\cjkit 
\textit{注:本简历使用 \href{https://www.latex-project.org/}{\LaTeX} 和 \XeTeX 编写, 隐私信息已脱敏。}
% \babelfont{rm}{FandolSong}

% \CJKfontslant{这是伪斜体中文。}

% 标题部分

\begin{center}
    \name{***}
\end{center}

% 简介部分
\section{个人简介}
\noindent \begin{tabularx}{\textwidth}{@{}L L L@{}}
    \faPhone \ 1********** & 
    \faEnvelope \ \href{mailto:***@***.com}{***@***.com} &
    \faGithub \ \href{https://gitee.com/***}{gitee.com/***} \\[0.2em]
    \faMapMarker \ 武汉市*** &
    \faGlobe \ \href{https://***.pages.dev}{***.pages.dev} &
    \faBlog \ \href{https://www.cnblogs.com/***}{cnblogs.com/***}
\end{tabularx}

\begin{itemize}
    \item 3年软件开发经验,熟练运用多种编程语言,具备跨平台开发能力
    \item 热衷于技术创新,善于解决复杂问题,追求高质量代码和高性能应用
    \item 熟悉软件开发流程: 需求\arrowsymbol 分析\arrowsymbol 设计\arrowsymbol 开发\arrowsymbol 测试\arrowsymbol 上线\arrowsymbol 维护及流程间的推动、完善与团队协作能力,能快速适应新环境并创造价值
\end{itemize}

\section{专业技能}
\begin{itemize}
    \item Core Dev: \href{https://baike.baidu.com/item/跨平台技术}{Multi-platform} \separator \href{https://baike.baidu.com/item/后端工程师}{Backend} \separator 
    \href{https://baike.baidu.com/item/敏捷开发模式/8395733}{Agile} \separator \href{https://www.ibm.com/cn-zh/topics/rest-apis}{RESTful API} \separator \href{https://baike.baidu.com/item/前端/5956545}{Frontend}
    \item Tech Stack: \href{https://spring.io/}{Spring} \separator \href{https://www.jetbrains.com/kotlin/}{Kotlin} \separator 
    \href{https://www.mysql.com/cn/}{Database} \separator \href{https://maven.apache.org/}{Maven} , \href{https://gradle.org/}{Gradle} \separator
    \href{https://vuejs.org/}{Vuejs}, \href{https://react.dev/}{Reactjs} \separator \href{https://www.typescriptlang.org/}{TypeScript}, \href{https://www.javascript.com/}{JavaScript} \separator 
    \href{https://nodejs.org/en/}{Nodejs}, \href{https://bun.sh/}{Bun} , \href{https://deno.com/}{Deno}, \href{https://webassembly.org/}{WebAssembly} \separator
    \href{https://vitejs.dev/}{Vite} \separator \href{https://developer.android.google.cn/jetpack/compose}{Jetpack Compose} \separator \href{https://flutter.dev/}{Flutter} \separator 
    \href{https://www.electronjs.org/}{Electron} \separator \href{https://uniapp.dcloud.net.cn/}{UniApp X} \separator \href{https://developer.huawei.com/consumer/cn/}{HarmonyOS App}
    \item IDE \& Env: \href{https://www.jetbrains.com/}{IntelliJ}, \href{https://code.visualstudio.com/}{VSCode}, \href{https://developer.android.google.cn/studio}{Android Studio}, \href{https://developer.harmonyos.com/cn/develop/deveco-studio}{DevEco Studio} \separator 
    \href{https://gradle.org/}{Gradle} \separator \href{https://www.cursor.com/}{Cursor} \separator \href{https://www.docker.com/}{Docker} \separator 
    \href{https://git-scm.com/}{Git}
    \item Trends Tech\footnote{集中于软件技术,包括但不限于编程语言、框架、工具、平台等,内容仅供参考}: \href{https://www.ibm.com/cn-zh/topics/large-language-models}{AI-LLM \footnote{ {\$36B equity investment , 2023 +111\% job postings difference 2022 }By mckinsey.com ; 个人:力大出奇迹,资源焚化器} } \separator 
    \href{https://www.rust-lang.org/}{Rust} \separator \href{https://www.bigdata.com/}{BigData} \separator \href{https://baike.baidu.com/item/web3.0/4873257}{Web 3.0}
\end{itemize}


\section{工作经验}
\company{XXXXX公司} 
\position{***}{***}
\daterange{*** - ***}
\begin{itemize}
    \item 参与核心产品开发,优化架构提升性能,实现复杂业务逻辑,保证代码质量
    \item 协作开发,参与并主导需求分析、代码审查,改进开发流程,解决技术难题,研究并应用新技术
\end{itemize}

\project{XXX系统}
\begin{itemize}
    \item 项目目标: 实时监控IT基础设施,快速识别并使相关人员快速处理故障,提高系统可靠性,生成符合标准的报告
        \begin{itemize}
            \item 负责工作: 项目规划与任务分工、需求分析及交付文档编写 \separator 开发技术选型与架构配置、后端代码开发、代码审查 \separator 测试计划制定及执行
            \item 项目细节与成效: 在团队高效沟通下,项目进展顺利。项目在前后端分离的情况下, 前端使用 \href{https://cn.vuejs.org/}{Vuejs} 基于 \href{https://ant.design/index-cn}{Ant Design} 来构建用户界面,
            后端采用 \href{https://spring.io/}{Spring} 系列项目来搭建应用模块, 使用 \href{https://docs.spring.io/spring-framework/reference/web-reactive.html}{Spring WebFlux} + \href{https://spring.io/projects/spring-data-r2dbc}{R2DBC} 处理生成高效的 \href{https://www.ibm.com/cn-zh/topics/rest-apis}{RESTful API}, 
            鉴权控制使用 \href{https://spring.io/projects/spring-security}{Spring Security} 同时为其他系统保留 \href{https://oauth.ac.cn/2/}{OAuth2.0 } 接口,使用 \href{https://www.mysql.com/cn/}{MySQL} 作为数据库底层,
            采用基于\href{https://www.omg.org/spec/CMMN}{CMMN} 和 \href{https://www.omg.org/spec/BPMN/2.0}{BPMN2.0} 规范的 \href{https://www.flowable.com/open-source/docs/}{Flowable} 引擎来实现工作流, 这里遗憾的是其开源版本只有 Block-IO JPA 实现,由于时间关系并未补充其 NoBlock-IO 实现, 故而求其次采用 \href{https://spring.io/projects/spring-data-jpa}{JPA} 。
            同时项目使用 \href{https://spring.io/projects/spring-ai}{Spring AI} 融合 \href{https://qwen.readthedocs.io/zh-cn/latest/index.html}{qwen} 等模型建立生成式对话。
            项目使用 \href{https://about.gitea.com/}{Gitea} 在内部网络搭建仓库进行协同开发和管理。
        \end{itemize}
    
\end{itemize}

\company{XXXXX公司} 
\position{***}{***}
\daterange{*** - ***}
\begin{itemize}
    \item 参与少儿编程的课程设计与研发,包括 Scratch 图形化编程及Lego WeDo 编程,支撑学员教学工作
    \item 提供计算机技术支持、软件开发支持
\end{itemize}

\company{XXXXX公司} 
\position{***}{***}
\daterange{*** - ***}
\begin{itemize}
    \item 参与项目系统设计,负责部分产品需求分析、接口设计和开发详设文档编写;根据产品需求,独立完成部分系统的后端开发及数据库表设计工作,完成开发中相关业务接口的定义及实现工作;
    \item 负责 Srpingboot、Spring cloud、Mybatis 应用开发及后端代码持续的维护和性能调优;完成直接上级交办的其他工作任务。
\end{itemize}

\project{XXX系统}
\daterange{*** - ***}
\begin{itemize}
    \item 项目目标:构建一个全面的医疗健康大数据体系,为医疗卫生领域提供定制化的大数据分析与数据治理平台。旨在为医护科研人员提供一个一站式的大数据共享平台,支持医疗监控、服务和协同应用。
        \begin{itemize}
            \item 核心工作内容: 模块设计与实现:负责接口和数据库设计,以及代码结构的设计与实现 \separator 涉及文档数据处理:使用 \href{https://poi.apache.org/}{Apache POI} 处理表格数据,执行数据校验和文件操作 \separator ORM框架应用:利用 Hibernate JPA 操纵管理数据库,实现业务逻辑,并开发实现相关API接口。
            \item 其他开发职责:数据编码与转换:进行数据编码、数据集转换处理和编辑 \separator 负责项目过程文档和产品文档的编写工作。
        \end{itemize}
\end{itemize}

\project{XXX系统}
\daterange{*** - ***}
\begin{itemize}
    \item 项目目标:构建国内领先的生物信息数据库,致力于成为科研领域内的专业综合网站,通过提供先进的数据检索、采集和处理工具,提升科研水平。
    \item 核心职责: 数据采集与处理: 利用 \href{https://webmagic.io/}{WebMagic} 配置爬虫,实现对国外科研知识数据库的检索与采集,将数据转换并存储为特定数据集 \separator 
            数据处理工具开发: 提供自定义脚本处理功能,支持科研人员进行数据分析,并实现数据的可视化展示 \separator 
            后端开发: 使用 \href{https://mybatis.org/mybatis-3/zh/index.html}{Mybatis} 及 \href{https://mybatis.plus/}{MybatisPlus} 开发数据功能模块,实现业务逻辑和数据操纵 \separator 
            API接口完善: 负责API接口的开发与维护,确保接口的稳定性和扩展性 \separator 代码质量管理: 承担代码审计和走读工作,确保代码质量,跟踪项目进度。
\end{itemize}

\project{XXX系统}
\daterange{*** - ***}
\begin{itemize}
    \item 项目概述:通过实施高效的流程管理,该项目实现了从研发到第三方的数据与工作流程互联互通。系统集成多项关键管理模块,从而能够有效提升整体工作效率和研发数据的有效性。
    \item 技术实现与职责:系统架构与开发:负责后端架构设计,确保系统功能完整性,包括数据库和接口设计 \separator 工作流引擎集成:使用  \href{https://github.com/snakerflow-starter/snakerflow-spring-boot-starter}{SnakerFlow} 引擎,实现流程自动化和管理 \separator 运用Maven进行项目依赖管理和分层开发 \separator
        基于\href{https://kkview.cn/zh-cn/index.html}{kkFileView} 定制开发,提升文件预览体验 \separator 接口开发:利用 \href{https://www.ssssssss.org/magic-api/}{Magic-API} 快速实现API接口 \separator
        权限管理:通过 \href{https://sa-token.cc/}{Sa-token} 实现RBAC模型,确保系统安全 \separator 文档数据处理:使用 \href{https://easyexcel.opensource.alibaba.com/}{EasyExcel} 优化Excel文件的数据处理。

\end{itemize}
   


\project{XXX系统}
\daterange{*** - ***}
\begin{itemize}
    \sloppy
    \item 负责心理测评系统的完整后端开发工作,确保系统功能与业务需求相符合 \separator 设计并实现了基于 SpringSecurity 的系统鉴权认证机制,采用RBAC模型确保用户权限的精确控制 \separator
    利用 \href{https://gitee.com/wupaas/easypoi}{easypoi} 工具高效处理表格数据转换,并进行数据校验,保障数据准确性和一致性 \separator 通过MybatisPlus工具快速搭建系统数据持久层,完成数据库模块的设计与实现 \separator 配置Nginx作为反向代理服务器,并部署Tomcat服务器,优化系统以应对高并发访问情况。
\end{itemize}

\section{教育背景}
\company{***} 
\position{*** \separator ***,***学位}{}
\daterange{*** - ***}

\section{兴趣爱好}
\begin{itemize}
    \item New Tech: \href{https://kotlinlang.org/}{Kotlin} \footnote{Reasons:对变量类型的标注与声明,代码的优雅艺术;扩展和匿名,Bring us a whole new World ! } \separator \href{https://www.rust-lang.org/}{Rust} \footnote{Reasons:因对内存的控制,使程序更安全} \separator \href{https://www.cursor.com/}{Cursor\footnote{Reasons: Every one is a super big team with AI-LLM ! }}
    \item Game: \href{https://lol.qq.com/}{League of Legends} \separator \href{https://store.steampowered.com/app/949230/Cities_Skylines_II/}{Cities: Skylines II} \separator \href{https://store.steampowered.com/app/289070/Sid_Meiers_Civilization_VI/}{Sid Meier's Civilization VI} \separator \href{https://store.steampowered.com/app/270880/American_Truck_Simulator/}{American Truck Simulator}
    \item Sports\footnote{遗憾:运动使我快乐,但常常很少做或没时间去做 - 时间都去打游戏了}: 多种球类运动 \separator 骑行 \separator 跑步 
    \item Theory: 交易成本经济学,积极心理学
\end{itemize}


\end{document}
